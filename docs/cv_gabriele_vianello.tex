%-------------------------
% Curriculum Vitae — Gabriele Vianello
% Tailored for: Bending Spoons Scholarship Application
% Compiled with: pdflatex or xelatex
%-------------------------
\documentclass[a4paper,11pt]{article}

%--- PACKAGES ---
\usepackage[utf8]{inputenc}
\usepackage[T1]{fontenc}
\usepackage{lmodern}
\usepackage[margin=1.8cm, top=1.6cm, bottom=1.6cm]{geometry}
\usepackage{titlesec}
\usepackage{enumitem}
\usepackage{hyperref}
\usepackage{xcolor}
\usepackage{fontawesome5}
\usepackage{tabularx}
\usepackage{multicol}
\usepackage{parskip}

%--- COLORS ---
\definecolor{primary}{HTML}{1A1A2E}
\definecolor{accent}{HTML}{0F3460}
\definecolor{link}{HTML}{1565C0}
\definecolor{subtle}{HTML}{555555}
\definecolor{rule}{HTML}{CCCCCC}

%--- HYPERLINK SETUP ---
\hypersetup{
    colorlinks=true,
    linkcolor=link,
    urlcolor=link,
    pdfauthor={Gabriele Vianello},
    pdftitle={CV Gabriele Vianello — Bending Spoons Scholarship},
    pdfsubject={Curriculum Vitae},
}

%--- SECTION FORMATTING ---
\titleformat{\section}
  {\Large\bfseries\color{primary}}
  {}{0em}{}
  [\vspace{-0.6em}\textcolor{rule}{\rule{\textwidth}{0.4pt}}]

\titlespacing*{\section}{0pt}{1.2em}{0.6em}

%--- CUSTOM COMMANDS ---
\newcommand{\cventry}[4]{%
  \textbf{#1} \hfill \textcolor{subtle}{\small #2} \\
  \textit{\color{accent}#3} \hfill \textcolor{subtle}{\small #4}
}

\newcommand{\cventrysingle}[2]{%
  \textbf{#1} \hfill \textcolor{subtle}{\small #2}
}

\newcommand{\projectentry}[4]{%
  \textbf{#1} — \textit{#2} \\
  {\small #3} \\
  {\footnotesize\textcolor{subtle}{Tecnologie: #4}}
}

%--- REMOVE PAGE NUMBER ---
\pagestyle{empty}

%--- DOCUMENT ---
\begin{document}

% ============================================================
% HEADER
% ============================================================
\begin{center}
  {\Huge\bfseries\color{primary} Gabriele Vianello} \\[0.5em]
  {\large\color{accent} AI \& Data Science Engineer}
  \vspace{0.6em}

  \faIcon{envelope}\;\href{mailto:vianello.tech@gmail.com}{vianello.tech@gmail.com} \quad
  \faIcon{github}\;\href{https://github.com/Vinello28}{Vinello28} \quad
  \faIcon{linkedin}\;\href{https://www.linkedin.com/in/gabriele-vianello-476a331a1}{LinkedIn} \quad
  \faIcon{map-marker-alt}\;Italia
\end{center}

\vspace{0.4em}

% ============================================================
% PROFILO
% ============================================================
\section{Profilo}
Studente magistrale in Ingegneria Informatica con specializzazione in \textbf{Intelligenza Artificiale e Data Science}.
Forte background in Machine Learning, Computer Vision a Data Engineering, affinato attraverso un percorso accademico d'eccellenza (Politecnico di Milano $\rightarrow$ Università Politecnica delle Marche) e oltre \textbf{14 progetti tecnici} che spaziano dalla classificazione semantica al forecasting finanziario.
Appassionato nell'integrazione tra algoritmi complessi e software scalabile, con particolare attenzione alla qualità del codice e alle architetture production-ready.

% ============================================================
% FORMAZIONE
% ============================================================
\section{Formazione}

\cventry{Laurea Magistrale in Ingegneria Informatica}{In corso}
  {Università Politecnica delle Marche (UnivPM)}{Ancona}
\begin{itemize}[leftmargin=1.2em, topsep=2pt, itemsep=1pt]
  \item Specializzazione in Artificial Intelligence \& Data Science.
  \item Corsi chiave: Deep Learning, Computer Vision, NLP, Data Mining, Network Science.
\end{itemize}

\vspace{0.5em}

\cventry{Laurea Triennale in Ingegneria dei Sistemi Informatici}{Completata}
  {Politecnico di Milano}{Cremona/Milano}
\begin{itemize}[leftmargin=1.2em, topsep=2pt, itemsep=1pt]
  \item Solida formazione in fondamenti dell'informatica, algoritmi, e architetture software.
\end{itemize}

\vspace{0.5em}

\cventry{Diploma di Perito Informatico}{Completato}
  {Istituto Tecnico Industriale V.\ Volterra}{Ancona}
\begin{itemize}[leftmargin=1.2em, topsep=2pt, itemsep=1pt]
  \item Indirizzo Informatica e Telecomunicazioni, specializzazione Informatica.
\end{itemize}

% ============================================================
% COMPETENZE TECNICHE
% ============================================================
\section{Competenze Tecniche}

\begin{tabularx}{\textwidth}{@{}l X@{}}
  \textbf{AI \& Data Science} & PyTorch, TensorFlow/Keras, Scikit-learn, Pandas, RASA, Ollama, NetworkX, ONNX, Spacy \\[0.3em]
  \textbf{Linguaggi}          & Python, C\#, TypeScript, Go, C, SQL, VB.NET \\[0.3em]
  \textbf{Backend \& DevOps}  & FastAPI, .NET Core, Docker, PostgreSQL, DuckDB, Qdrant, GitHub Actions, BullMQ \\[0.3em]
  \textbf{Frontend}           & React, Vite, TailwindCSS, Framer Motion, Syncfusion \\[0.3em]
  \textbf{Metodologie}        & SWE, ETL Pipelines, RAG, Containerization, CI/CD, Agile \\
\end{tabularx}

% ============================================================
% PROGETTI SELEZIONATI
% ============================================================
\section{Progetti Selezionati}

\projectentry{Boost-a-Model}{AI Engineer — Progetto Universitario}
{Training, fine-tuning e validazione comparativa di architetture \textbf{Transformer e CNN} per task di Visual Servoing. Utilizzo intensivo di PyTorch/TensorFlow per l'addestramento su GPU con analisi approfondita delle metriche di Computer Vision.}
{PyTorch, TensorFlow, CUDA, Deep Learning}

\vspace{0.6em}

\projectentry{Pack-a-Punch}{AI Engineer}
{Microservizio di \textbf{classificazione semantica} con pipeline di training e inference server dockerizzato, incluso export in formato ONNX per deployment ottimizzato.}
{CUDA, Python, PyTorch, Docker, ONNX}

\vspace{0.6em}

\projectentry{Bandolero}{AI Engineer \& Software Architect}
{Sistema \textbf{RAG (Retrieval-Augmented Generation)} che sfrutta Small Language Models per QA su documenti PDF in ambiente multi-utente. Implementazione di vector store con Qdrant e retrieval contestuale.}
{Python, Golang, Qdrant, Docker, Ollama}

\vspace{0.6em}

\projectentry{Dora-the-Data-Explorer}{Data Scientist \& ML Engineer — Team Project}
{Pipeline ML completa: da EDA avanzata a Feature Engineering e Model Selection (Random Forest, XGBoost). Ottimizzazione rigorosa di Accuracy, F1-Score e ROC-AUC tramite hyperparameter tuning.}
{Python, Scikit-learn, XGBoost, Pandas}

\vspace{0.6em}

\projectentry{GoldenHour}{Data Scientist — Team Project}
{Forecasting predittivo su \textbf{serie storiche finanziarie} (metalli preziosi). Implementazione di modelli ARIMA/SARIMA e LSTM con validazione tramite backtesting e valutazione RMSE/MAE.}
{Python, TensorFlow/Keras, ARIMA, Time Series}

\vspace{0.6em}

\projectentry{Open-Data-Chunker}{Data Engineer}
{Pipeline \textbf{ETL ad alte prestazioni} per processare massicci file XML OpenData. Focus su efficienza algoritmica, gestione della memoria e parallelizzazione con storage in formato Parquet.}
{Python, Docker, DuckDB, Parquet}

\vspace{0.6em}

\projectentry{teampa-25/core}{Software Engineer — Team Project}
{Soluzione backend per una piattaforma di \textbf{inferenza AI} su modelli di Computer Vision, con gestione asincrona dei task tramite code di messaggi.}
{TypeScript, Python, Docker, FastAPI, BullMQ}

% ============================================================
% ALTRI PROGETTI RILEVANTI
% ============================================================
\section{Altri Progetti}

\begin{multicols}{2}
\begin{itemize}[leftmargin=1.2em, topsep=2pt, itemsep=3pt]
  \item \textbf{Uncounsciously-Sincere-Bot} — Chatbot conversazionale con RASA, NLU custom e Docker.
  \item \textbf{Faboulous-Interpretr} — Interpretazione semantica del testo e text mining sperimentale.
  \item \textbf{ASP.NET Enterprise Solutions} — Soluzioni enterprise in C\#/.NET con crittografia custom e automazione documentale.
\end{itemize}
\end{multicols}

% ============================================================
% SOFT SKILLS E INTERESSI
% ============================================================
\section{Soft Skills \& Interessi}

\begin{itemize}[leftmargin=1.2em, topsep=2pt, itemsep=2pt]
  \item \textbf{Problem Solving} — Approccio analitico e orientato ai dati, dimostrato attraverso progetti che spaziano dall'ottimizzazione di pipeline ETL al fine-tuning di modelli neurali.
  \item \textbf{Lavoro in Team} — Esperienza consolidata in progetti collaborativi universitari (7+ team project), con ruoli di SW e Data Engineer.
  \item \textbf{Apprendimento Continuo} — Costante aggiornamento sulle tecnologie emergenti (LLM, vLLM, Podman) e del modo più efficiente ed efficace per implementarle.
  \item \textbf{Comunicazione Tecnica} — Capacità di documentare e presentare soluzioni complesse in modo chiaro e accessibile.
\end{itemize}

\vspace{1.2em}
\begin{center}
  \small\textcolor{subtle}{Autorizzo il trattamento dei miei dati personali ai sensi del D.Lgs. 196/2003 e del GDPR (Reg. UE 2016/679).}
\end{center}

\end{document}
